\documentclass[a4paper]{article}

\usepackage{amsmath}
\usepackage{amssymb}

\title{Derivative of a Continued Fraction}
\author{Edward Abel-Guobadia}
\date{6-17-2023}

\begin{document}
\maketitle
\newpage
\section*{Definitions}
\begin{align}
    g(x) &= \cfrac{1}{f(x) + \cfrac{1}{f(x) + \cfrac{1}{f(x) + \ddots}}}\\
    g^\prime(x) &= ?
\end{align}
if y = g(x) and x = f(x)
then:
\begin{align}
    g(x) &= \cfrac{1}{f(x) + \cfrac{1}{f(x) + \cfrac{1}{f(x) + \ddots}}}
    \equiv y = \cfrac{1}{x + \cfrac{1}{x + \cfrac{1}{x + \ddots}}}\\
    \therefore g^\prime(x) &= y^\prime
\end{align}
since
\begin{align}
    y = \cfrac{1}{x + \cfrac{1}{x + \cfrac{1}{x + \ddots}}}
\end{align}
then
\begin{align}
    y = \frac{1}{x + y} 
\end{align}

\section*{Algebra}
since
\begin{align}
    y &= \frac{1}{x + y}\\
    \therefore 1 &= y(x + y)\\
    0 &= y^2 + xy - 1
\end{align}
by the quadratic formula we find y to be:
\begin{align}
    y = \frac{-x \pm\sqrt{x^2+4}}{2}
\end{align}
we will define g to be the negative part of y and h to be the positive part:
\begin{align}
    g = \frac{-\sqrt{x^2+4} - x}{2} \quad\text{and}\quad h = \frac{\sqrt{x^2+4} - x}{2}
\end{align}
\section*{Calculus}
\begin{align*}
    g^\prime = \frac{-1 + \frac{x}{\sqrt{x^2+4}}}{2} \quad\text{and}\quad h^\prime = \frac{-1 - \frac{x}{\sqrt{x^2+4}}}{2}
\end{align*}
since g is just the - part of y and h is just the + part. We can see that for some point y = g or y = h
\begin{align}
    (y^\prime-g^\prime)(y^\prime-h^\prime) &= 0\\
    (y^\prime)^2 - h^\prime y^\prime - g^\prime y^\prime + g^\prime h^\prime &= 0\\
    \therefore (y^\prime)^2 - (h^\prime + g^\prime)y^\prime + g^\prime h^\prime &= 0\\
    \text{since}\quad h^\prime + g^\prime = -1 \quad\text{and}\quad g^\prime h^\prime = \frac{1}{4}-\frac{x^2}{4(x^2+4)}\\\\
    \text{we arrive at}\qquad (y^\prime)^2+y^\prime-\left(\frac{x^2}{4(x^2+4)}-\frac{1}{4}\right) &= 0
\end{align}
we can simplify the last term:
\begin{equation}
    \left(\frac{x^2}{4(x^2+4)}-\frac{1}{4}\right) = \frac{-1}{x^2+4}
\end{equation}
giving us
\begin{equation}
    (y^\prime)^2+y^\prime+\frac{1}{x^2+4} = 0
\end{equation}
we will introduce a new variable u and set it to the last term:
\begin{align}
    u &= \frac{1}{x^2+4}
\end{align}
we can immediately see:
\begin{align}
    (y^\prime)^2+y^\prime+u&=0\\
    (y^\prime)^2+y^\prime&=-u\\
    y^\prime(y^\prime+1)&=-u
\end{align}
Finnaly we arrive at:
\begin{equation}
    y^\prime = \frac{u}{1+y^\prime}
\end{equation}
in expanded form we see:
\begin{equation}
    y^\prime=\cfrac{u}{1+\cfrac{u}{1+\cfrac{u}{1+\ddots}}}
\end{equation}

\section*{Are u important... yes :)}
From the initial definitions of g(x) and f(x) I think I can infer that if:
\begin{align}
    g(x) &= \cfrac{1}{f(x)+\cfrac{1}{f(x)+\cfrac{1}{f(x)+\ddots}}}\\
    \text{then}\\
    g^\prime(x)&=\cfrac{u}{f^\prime(x)+\cfrac{u}{f^\prime(x)+\cfrac{u}{f^\prime(x) + \ddots}}}\\
    \text{for some u}
\end{align}
My question is what determines u? Is there some way to calculate u? Some algorithm we can follow that gives us u?
I tried lots of strange and suspicious math, but my tactics could not find a system for determining u.
On my quest for u I found some strange results that I have written in my notebook, but I suspect they are unworthy to be seen by anyone
Anyway thats all I have for u... I might try integrating a continued fraction(c.f.) or a backward continued fraction(b.c.f.)
Or maybe a c.f. of derivatives Or what about a c.f. of integrals!?
Also I think there might be strange connection between c.f's and generating functions, but I have done no math on the matter, just a hunch I have

\end{document}