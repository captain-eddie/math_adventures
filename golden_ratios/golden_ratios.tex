
\documentclass[a4paper]{article}

\usepackage{amsmath}
\usepackage{amssymb}

\begin{document}

\section*{A peculiar continued fraction}
\begin{align*}
    f(x) &= \cfrac{1}{x + \cfrac{1}{x + \cfrac{1}{x + \ddots}}} \\
    \intertext{now consider f(x) - 1, in other words} \\
    f(x) &= 1 + \cfrac{1}{x + \cfrac{1}{x + \cfrac{1}{x + \ddots}}} \quad \text{for x = 1 we have ...}\\
    f(1) &= 1 + \cfrac{1}{1 + \cfrac{1}{1 + \cfrac{1}{1 + \ddots}}}\\
    \intertext{I will show that this continued fraction "converges" and surprisingly "converges" to the golden ratio}\\
    \intertext{the "compact" form of the peculiar continued fraction, is found by...}\\
    f(x) &= \cfrac{1}{x + \cfrac{1}{x + \cfrac{1}{x + \ddots}}},\quad y = f(x)\implies y = \cfrac{1}{x + \cfrac{1}{x + \cfrac{1}{x + \ddots}}}\\
    \intertext{Therefore...} y &= \frac{1}{x + y} \equiv y = \cfrac{1}{x + \cfrac{1}{x + \cfrac{1}{x + \ddots}}}\\
    y &= \frac{1}{x + y} \implies y(x + y) = 1 \implies y^2 + xy = 1\\
    \intertext{A nice compact quadratic polynomial that can be expanded to the pecuilar continued fraction}\\
    \intertext{Following the same algebraic manipulation it is easy to see that...}
    f(x) &= 1 + \cfrac{1}{x + \cfrac{1}{x + \cfrac{1}{x + \ddots}}} \intertext{when} \quad f(x) &= y \quad\text{and}\quad x = 1\\
    \implies y &= 1 + \cfrac{1}{1 + \cfrac{1}{1 + \cfrac{1}{1 + \ddots}}} \quad\text{is equivalent to} \quad y = 1 + \frac{1}{y}\\
    \intertext{Then be the same algebraic manipulation we arrive at a very similar quadratic equation:}\\
    0 &= y^2 - y - 1\\
    \intertext{From the quadratic equation we can find the roots:}\\
    y &= \frac{1 \pm {\sqrt{5}}}{2}\\
    \intertext{and strangely $\phi$, the golden ratio, is the positive root of the polynomial..}\\
    \phi &= \frac{1 + \sqrt{5}}{2}
\end{align*}

\section*{A doubly pecuilar nested radical}
\begin{align*}
    \intertext{Recall the polynomial whose positieve and negative roots are the positive and negative golden ratio, respectively}\\
    0 &= y^2-y-1\\
    \intertext{if we add y and 1 to the equation and take the sqaure root we arrrive at a interesting radical equation}\\
    \implies y &= \sqrt{1 + y}\\
    \intertext{Then if we expand we arrive at cool nested radical}\\
    y &= \sqrt{1 + \sqrt{1 + \sqrt{1 + \dots}}}\\
    \intertext{since} y &= \frac{1 + \sqrt{5}}{2} \intertext{and} \phi &= \frac{1 + \sqrt{5}}{2}\\
    \intertext{then} \phi &= \sqrt{1 + \sqrt{1 + \sqrt{1 + \dots}}}\\
    \intertext{and also since} y &= \sqrt{1 + y}\\
    \intertext{then} \phi &=  \sqrt{1 + \phi}\\
    \intertext{A very strange result, it doesn't make sense of a number to be equal to the sqaure root of itself plus 1, $2 \neq \sqrt{1+2}$}\\
    \intertext{But, i have arrived at an interesting result...}
\end{align*}

\section*{Why is phi, the golden ratio important?}
\begin{align*}
    \intertext{so far I have shown that only phi is a solution to the following three equations}\\
    0 &= y^2-y-1\\
    y&=1+\frac{1}{y}\\
    y&=\sqrt{1+y}\\
    \intertext{Only when $y=\pm\phi$ are the above equations true, since the nested radical and contiued fraction forms flow from the polynomial form and since there are only 2 solutions to the polynomial form $y=\pm\phi$, then $\pm\phi$ is the only solution to the nested radical and coninuted fraction form}\\
    \intertext{Then this means thats the answer to the question that titles this section, $\phi$ is important because it is the only number that solves that polynomial and subsequently solves those equations.}\\
    \intertext{But why limit ourselves to just this arbitrary polynomial? What's so special about $0 = y^2-y-1$ other than phi being its only solution?}\\
    \intertext{I will show that there is another equally peculiar polynomial with an accompanying set of equations with equally irrational solutions, in fact I will show that there are precisely countably infine sets of equations that satisfy the mystical properties of $\phi$}\\
    \intertext{Consider the polynomial:}\\
    0 &= y^2-ny-1 \text{,}\quad n \in \mathbb{N^+}\\
    \intertext{The solution to this quadratic is:}\\
    y&=\frac{n\pm\sqrt{n^2+4}}{2}\\
    \intertext{I will show that for any positive natural number this solution is always irrational and I will show that this solution is the only soltuion to a certain set of peculiar equations... just like the golden ratio}\\
    \\
\end{align*}

\end{document}
